


\section{The Reverse Differential}

In order to apply gradient descent to our trainable parameters, we obviously have a need to compute various gradients of the cost function which is essentially a large functional composition.  Computing intermediate gradients along this computation doesn't make sense mathematically as stated.  However, the usual exterior derivative works very well in this context.  However, since we would like to vectorize this process, the exterior derivative falls short for our implementation purposes.  This leads us to a related form of differentiation, namely, the reverse derivative.  We give here a brief exposition of the reverse differential in the setting of Riemannian geometry, and then use Euclidean spaces as our examples.  C.f.,  \cite{barendregt1988introduction}, \cite{blute2009cartesian}, \cite{cockett2019reverse}, \cite{cruttwell2022categorical}, \cite{fong2019backprop}, \cite{gavranovic2019compositional},  \cite{mac2013categories}, \cite{mak2020differential}, \cite{selinger2010survey}, \cite{shiebler2021category}, \cite{wengert1964simple}.

We first recall the definition of the exterior derivative between smooth manifolds.
\begin{defn}
	Suppose $M,N$ are smooth manifolds and $f:M\to N$ is smooth.  Then for $p\in M$, the (exterior) differential of $f$ at $p$, denoted $df_p$, is the linear map
	$$df_p:T_pM\to T_{f(p)}N$$,
	such that for any $\xi\in T_pM$ and any $g\in C^\infty(N)$, we have that
	$$df_p(\xi)[g]=\xi[g\circ f].$$
\end{defn}

\begin{ex}
	Suppose $f:\R^n\to\R^m$ is smooth with coordinates $(x^j)$ on $\R^n$ and coordinates $(y^j)$ on $\R^m$.  Then at a point $p\in\R^n$, we have the differential in coordinates
	$$df_p=\frac{y^i\circ f}{\partial x^j}(p)\rest{dx^j}_p\otimes\rest{\frac{\partial}{\partial y^i}}_{f(p)}.$$
	In matrix form, we have the Jacobian representation of $df_p$, denoted $Jf_p\in\R^{m\times n}$, given by
	$$Jf_p=\begin{bmatrix}
		\rest{\frac{\partial f^1}{\partial x^1}}_p&\cdots&\rest{\frac{\partial f^1}{\partial x^n}}_p\\
		\rest{\frac{\partial f^2}{\partial x^1}}_p&\cdots&\rest{\frac{\partial f^2}{\partial x^n}}_p\\
		\vdots&\ddots&\vdots\\
		\rest{\frac{\partial f^m}{\partial x^1}}_p&\cdots&\rest{\frac{\partial f^m}{\partial x^n}}_p
	\end{bmatrix},$$
	where $f^i:=y^i\circ f.$
	
	Moreover, for any fixed $p\in\R^n$, we may identify $\R^n$ with the tangent space $T_p\R^n$ via
	$$v=(v^1,...,v^n)\in\R^n\leftrightsquigarrow \vec{v}=v^j\rest{\frac{\partial}{\partial x^j}}_p\in T_p\R^n.$$
	
	It then follows that
	\begin{align*}
		df_p(\vec{v})&=v^j\rest{\frac{\partial f^i}{\partial x^j}}_p\rest{\frac{\partial}{\partial y^i}}_{f(p)}\\
		&\leftrightsquigarrow\left(v^j\rest{\frac{\partial f^1}{\partial x^j}}_p,...,v^j\rest{\frac{\partial f^m}{\partial x^j}}_p\right)\\
		&=Jf_pv
	\end{align*}
\end{ex}

\begin{defn}\label{def:reverseDifferential}
	Suppose $(M,g)$ and $(N,h)$ are Riemannian manifolds and suppose $f:M\to N$ is smooth.  Then for $p\in M$, the reverse differential, denoted $rf_p$, is the linear map
	$$rf_p:T_{f(p)}M\to T_pM$$
	such that for any $\xi\in T_pM$ and any $\zeta\in T_{f(p)} N$, the following equality holds
	$$g(rf_p(\zeta),\xi)=h(\zeta, df_p(\xi)).$$
\end{defn}

\begin{ex}
	Suppose $f:\R^n\to\R^m$ is smooth with coordinates $(x^j)$ on $\R^n$ and coordinates $(y^j)$ on $\R^m$.  Then at a point $p\in\R^n$, we have the reverse differential in coordinates
	$$rf_p=\sum_{i=1}^m\sum_{j=1}^n\rest{\frac{\partial f^i}{\partial x^j}}_p\rest{dy^i}_{f(p)}\otimes\rest{\frac{\partial}{\partial x^j}}_p,$$
	where $f^i:=y^i\circ f$.
	
	In matrix form, we have the Jacobian representation of $rf_p$, denoted $J^Tf_p\in\R^{n\times m}$, given by
	$$J^Tf_p=\begin{bmatrix}
		\rest{\frac{\partial f^1}{\partial x^1}}_p&\cdots&\rest{\frac{\partial f^m}{\partial x^1}}_p\\
		\rest{\frac{\partial f^1}{\partial x^2}}_p&\cdots&\rest{\frac{\partial f^m}{\partial x^2}}_p\\
		\vdots&\ddots&\vdots\\
		\rest{\frac{\partial f^1}{\partial x^n}}_p&\cdots&\rest{\frac{\partial f^m}{\partial x^n}}_p.
	\end{bmatrix}$$
	
	Moreover, for $w\in\R^m\leftrightsquigarrow\vec{w}\in T_{f(p)}\R^m$ and $v\in\R^n\leftrightsquigarrow\vec{v}\in T_p\R^n$, it follows that
	\begin{align*}
		\ip{rf_p(\vec{w}),\vec{v}}_{T_p\R^n}&=\ip{\vec{w},df_p(\vec{v})}_{T_{f(p)}\R^m}\\
		&=\ip{w,Jf_p(v)}_{\R^m}\\
		&=\ip{J^Tf_p(w),v}_{\R^n},
	\end{align*}
	 and hence that
	 $$rf_p(\vec{w})=J^Tf_p(w).$$
\end{ex}


\begin{prop}
	Suppose we have the compositional diagram
	\begin{equation*}
		\begin{tikzcd}
			(M,g)
			\arrow[r, "\phi"]
			&(N, h)
			\arrow[r, "\psi"]
			&(Q,k)
		\end{tikzcd}
	\end{equation*}
	and we let $f:=\psi\circ\phi:(M,g)\to(Q,k).$
	Then for any $p\in M$, the reverse derivative satisfies
	$$rf_p=r\phi_p\circ r\psi_{\phi(p)}.$$
\end{prop}

\begin{proof}
Fix $p\in M$, and let $\xi\in T_pM$ and $\zeta\in T_{f(p)}Q$.  Then we have that
\begin{align*}
	g(rf_p(\zeta),\xi)&=k(\zeta,df_p(\xi))\\
	&=k(\zeta,d\psi_{\phi(p)}\circ d\phi_p(\xi))\\
	&=h(r\psi_{\phi(p)}(\zeta),d\phi_p(\xi))\\
	&=g(r\phi_p(r\psi_{\phi(p)}(\zeta)),\xi)\\
	&=g(r\phi_p\circ r\psi_{\phi(p)}(\zeta),\xi),
\end{align*}
as desired.
\end{proof}

\TOX{
The following needs to be refined further still.}

\begin{ex}
	Suppose $f:(\R^{m\times n}, (X^i_j), F)\to(\R, (t), \delta)$ is smooth, where $F$ is the Frobenius inner product.  Suppose $v\in T_P\R^{m\times n}\leftrightsquigarrow V\in\R^{m\times n}$ are represented via
	$$v=v^i_j\rest{\frac{\partial}{\partial X^i_j}}_P\leftrightsquigarrow V=\begin{bmatrix}
		v^i_j
	\end{bmatrix},$$
	and in coordinates, we have that
	$$df_P=\rest{\frac{\partial f}{\partial X^i_j}}_P\rest{dX^i_j}_P.$$
	The matrix-Jacobian-representation of $f$ at $P$, denoted $Jf_P\in\R^{m\times n}$ is given by
	$$Jf_P=\begin{bmatrix}
		\rest{\frac{\partial f}{\partial X^1_1}}_P&\cdots &\rest{\frac{\partial f}{\partial X^1_n}}_P\\
		\rest{\frac{\partial f}{\partial X^2_1}}_P&\cdots&\rest{\frac{\partial f}{\partial X^2_n}}_P\\
		\vdots&\ddots&\vdots\\
		\rest{\frac{\partial f}{\partial X^m_1}}_P&\cdots&\rest{\frac{\partial f}{\partial X^m_n}}_P
	\end{bmatrix}.$$
	
	It then follows that
	\begin{align*}
		df_P(v)&=v^i_j\rest{\frac{\partial f}{\partial X^i_j}}_P\\
		&=\ip{V,Jf_P}_{F(m,n)}.
	\end{align*}
	Similarly, if $\tau\in\R\leftrightsquigarrow\vec{\tau}=\tau\rest{\frac{d}{dt}}_{f(P)}\in T_{f(P)}\R$, we see the reverse differential given in coordinates
	$$rf_P=\sum_{i=1}^m\sum_{j=1}^n\rest{\frac{\partial f}{\partial X^i_j}}_P\rest{dt}_P\otimes\rest{\frac{\partial}{\partial X^i_j}}_{f(P)},$$
	evaluates to
	$$rf_p(\vec{\tau})=\tau\sum_{i=1}^m\sum_{j=1}^n\rest{\frac{\partial f}{\partial X^i_j}}_P\rest{\frac{\partial}{\partial X^i_j}}_{f(P)},$$
	and hence that
	\begin{align*}
		\ip{rf_P(\vec{\tau}),v}_{T_P\R^{m\times n}}&=\ip{\vec{\tau},df_P(v)}_{T_{f(P)}\R}\\
		&=\tau df_P(v)\\
		&=\tau\ip{V, Jf_P}_{F(m,n)}
	\end{align*}
\end{ex}






\begin{lem}
	Suppose $f:\R^{n\times m}\to\R^k$, and for $P\in\R^{n\times m}$, let $R=rf_P$.  Then $R\in\R^k{_n}{^m}$ is rank $(1,2)$-tensor written in coordinates as
	$$R=R_i{^\mu}{_\nu}\frac{\partial}{\partial X^\mu_\nu}\otimes dx^i,$$
	and the components is given by
	$$R_i{^\mu}{_\nu}=\frac{\partial f^i}{\partial X^\nu_\mu}$$
\end{lem}

\begin{proof}
	Considering the basis vectors $\frac{\partial}{\partial X^\nu_\mu}\in T_P\R^{n\times m}$ and $\frac{\partial}{\partial x^i}\in T_{f(P)}\R^k$ we have that
	\begin{align*}
		R_i{^\mu}{_\nu}&=\ip{R\left(\frac{\partial}{\partial x^i}\right), \frac{\partial}{\partial X^\nu_\mu}}_F\\
		&=\ip{\frac{\partial}{\partial x^i},df_P\left(\frac{\partial}{\partial X^\nu_\mu}\right)}_{\R^k}\\
		&=\ip{\frac{\partial}{\partial x^i},\frac{\partial f^\alpha}{\partial X^\nu_\mu}\frac{\partial}{\partial x^\alpha}}_{\R^k}\\
		&=\delta_{i\alpha}\frac{\partial f^\alpha}{\partial X^\nu_\mu},
	\end{align*}
	as desired.
\end{proof}





