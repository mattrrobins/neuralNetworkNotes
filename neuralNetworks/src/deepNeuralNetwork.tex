


\section{Deep Neural Networks}

In this section we discuss a general ``deep'' neural network, which consist of $L$ layers.  That is, we have a network of the form:
\begin{align*}
	&\underbrace{\begin{bmatrix}
		x^1\\
		\vdots\\
		x^{s_0}
		\end{bmatrix}}_{\text{Layer } 0}
	\layerfctn{\lay{\varphi}{1}}\underbrace{\begin{bmatrix}
			\lay{z}{1}{^1}\\
			\vdots\\
			\lay{z}{1}{^{s_1}}
			\end{bmatrix}\layerfctn{\lay{g}{1}}
			\begin{bmatrix}
			\lay{a}{1}{^1}\\
			\vdots\\
			\lay{a}{1}{^{s_1}}
			\end{bmatrix}}_{\text{Layer } 1}\layerfctn{\lay{\varphi}{2}}
	\underbrace{\begin{bmatrix}
		\lay{z}{2}{^1}\\
		\vdots\\
		\lay{z}{2}{^{s_2}}
		\end{bmatrix}\layerfctn{\lay{g}{2}}
		\begin{bmatrix}
			\lay{a}{2}{^1}\\
			\vdots\\
			\lay{a}{2}{^{s_2}}
		\end{bmatrix}}_{\text{Layer } 2}\layerfctn{\lay{\varphi}{3}}\cdots\\
	&\cdots\layerfctn{\lay{\varphi}{L-1}}
	\underbrace{\begin{bmatrix}
		\lay{z}{L-1}{^1}\\
		\vdots\\
		\lay{z}{L-1}{^{s_{L-1}}}
	\end{bmatrix}\layerfctn{\lay{g}{L-1}}\begin{bmatrix}
		\lay{a}{L-1}{^1}\\
		\vdots\\
		\lay{a}{L-1}{^{s_{L-1}}}
	\end{bmatrix}}_{\text{Layer }L-1}\layerfctn{\lay{\varphi}{L}}\underbrace{\begin{bmatrix}
		\lay{z}{L}{^1}\\
		\vdots\\
		\lay{z}{L}{^{s_L}}
	\end{bmatrix}\layerfctn{\lay{g}{L}}\begin{bmatrix}
		\lay{a}{L}{^1}\\
		\vdots\\
		\lay{a}{L}{^{s_L}}
	\end{bmatrix}}_{\text{Layer }L}\layerfctn{=}\begin{bmatrix}
		\hat{y}^1\\
		\vdots\\
		\hat{y}^{s_L}
	\end{bmatrix},
\end{align*}
where
$$s_\ell :=\text{ the number of nodes in layer-$\ell$},$$
$$\lay{\varphi}{\ell}:\R^{s_{\ell-1}}\to\R^{s_\ell},\qquad \lay{\varphi}{\ell}(\xi)=\lay{W}{\ell}\xi+\lay{b}{\ell},\qquad \lay{W}{\ell}\in\R^{s_\ell\times s_{\ell-1}},b\in\R^{s_\ell},$$
and
$$\lay{g}{\ell}:\R^{s_\ell}\to\R^{s_\ell},$$
is a broadcasted activation function determined by the layer-$\ell$.

As with a shallow network, our functional composition to obtain $\lay{a}{L}$ is known as forward propagation.

\subsection{Backpropagation}
As the general derivation for backpropagation can be easily (if not tediously) generalized from \cref{sec:backPropDerivation} using induction, we give the general outline for computational purposes.

Let $\L:\R^{s_L}\times\R^{s_L}\to\R$ be a generic loss function, and suppose our cost function is given by the usual
$$\J(W,b)=\frac{1}{n}\sum_{j=1}^n\L(\hat{y}_j,y_j).$$
Then from previous computations, we have the following gradients for any $\ell\in\{1,2,...,L\}$, that
\begin{align*}
	\frac{\partial\J}{\partial\lay{W}{\ell}}&=\frac{1}{n}\lay{\delta}{\ell}\lay{a}{\ell-1}{^T}\\
	\frac{\partial\J}{\partial\lay{b}{\ell}}&=\frac{1}{n}\sum_{j=1}^n\lay{\delta}{\ell}{_j}
\end{align*}
where we impose the notation of
$$\lay{a}{0}:=x.$$
So we need only give a full characterization of $\lay{\delta}{\ell}.$. To this end, we define recursively starting at layer-$L$ by
\begin{align*}
	\lay{\delta}{L}{^T}&:=d(\L_y)_{\lay{a}{L}}\cdot d\lay{g}{L}_{\lay{z}{L}},\\
	\lay{\delta}{L-1}{^T}&:=\lay{\delta}{L}{^T}\cdot\lay{W}{L}\cdot d\lay{g}{L-1}_{\lay{z}{L-1}},\\
	&\vdots\\
	\lay{\delta}{\ell}{^T}&:=\lay{\delta}{\ell+1}{^T}\lay{W}{\ell+1}d\lay{g}{\ell}_{\lay{z}{\ell}},\\
	&\vdots\\
	\lay{\delta}{1}{^T}&:=\lay{\delta}{2}{^T}\lay{W}{2}d\lay{g}{1}_{\lay{z}{1}},
\end{align*}
as desired.



\subsubsection{Vectorization in Python}
We implement a neural network with an arbitrary number of layers and nodes, with the $\relu$ function as the activator on all hidden nodes and the sigmoid function on the output layer for binary classification with the log-loss function.

\lstinputlisting[lastline=300]{src/py/deepNeuralNetwork.py}













