


\section{Deep Neural Networks}

In this section we discuss a general ``deep'' neural network, which consist of $L$ layers.  That is, we have a network of the form:
\begin{align*}
	&\underbrace{\begin{bmatrix}
		x^1\\
		\vdots\\
		x^{m_0}
		\end{bmatrix}}_{\text{Layer } 0}
	\layerfctn{\lay{\varphi}{1}}\underbrace{\begin{bmatrix}
			\lay{z}{1}{^1}\\
			\vdots\\
			\lay{z}{1}{^{m_1}}
			\end{bmatrix}\layerfctn{\lay{g}{1}}
			\begin{bmatrix}
			\lay{a}{1}{^1}\\
			\vdots\\
			\lay{a}{1}{^{m_1}}
			\end{bmatrix}}_{\text{Layer } 1}\layerfctn{\lay{\varphi}{2}}
	\underbrace{\begin{bmatrix}
		\lay{z}{2}{^1}\\
		\vdots\\
		\lay{z}{2}{^{m_2}}
		\end{bmatrix}\layerfctn{\lay{g}{2}}
		\begin{bmatrix}
			\lay{a}{2}{^1}\\
			\vdots\\
			\lay{a}{2}{^{m_2}}
		\end{bmatrix}}_{\text{Layer } 2}\layerfctn{\lay{\varphi}{3}}\cdots\\
	&\cdots\layerfctn{\lay{\varphi}{L-1}}
	\underbrace{\begin{bmatrix}
		\lay{z}{L-1}{^1}\\
		\vdots\\
		\lay{z}{L-1}{^{m_{L-1}}}
	\end{bmatrix}\layerfctn{\lay{g}{L-1}}\begin{bmatrix}
		\lay{a}{L-1}{^1}\\
		\vdots\\
		\lay{a}{L-1}{^{m_{L-1}}}
	\end{bmatrix}}_{\text{Layer }L-1}\layerfctn{\lay{\varphi}{L}}\underbrace{\begin{bmatrix}
		\lay{z}{L}{^1}\\
		\vdots\\
		\lay{z}{L}{^{m_L}}
	\end{bmatrix}\layerfctn{\lay{g}{L}}\begin{bmatrix}
		\lay{a}{L}{^1}\\
		\vdots\\
		\lay{a}{L}{^{m_L}}
	\end{bmatrix}}_{\text{Layer }L}\layerfctn{=}\begin{bmatrix}
		\hat{y}^1\\
		\vdots\\
		\hat{y}^{m_L}
	\end{bmatrix},
\end{align*}
where
$$m_\ell :=\text{ the number of nodes in layer-$\ell$},$$
$$\lay{\varphi}{\ell}:\R^{m_{\ell-1}}\to\R^{m_\ell},\qquad \lay{\varphi}{\ell}(\xi)=\lay{W}{\ell}\xi+\lay{b}{\ell},\qquad \lay{W}{\ell}\in\R^{m_\ell\times m_{\ell-1}},b\in\R^{m_\ell},$$
and
$$\lay{g}{\ell}:\R^{m_\ell}\to\R^{m_\ell},$$
is a broadcasted activation function determined by the layer-$\ell$.

As with a shallow network, our functional composition to obtain $\lay{a}{L}$ is known as forward propagation.

\subsection{Backward Propagation}
As the general derivation for backpropagation can be easily (if not tediously) generalized from \cref{sec:backPropDerivation} using induction, we give the general outline for computational purposes.

Let $\L:\R^{m_L}\times\R^{m_L}\to\R$ be a generic loss function, and suppose our cost function is given by the usual
$$\J(W,b)=\frac{1}{n}\sum_{j=1}^n\L(\hat{y}_j,y_j).$$
Then from previous computations, we have the following gradients for any $\ell\in\{1,2,...,L\}$, that
\begin{align*}
	\frac{\partial\J}{\partial\lay{W}{\ell}}&=\frac{1}{n}\lay{\delta}{\ell}\lay{a}{\ell-1}{^T}\\
	\frac{\partial\J}{\partial\lay{b}{\ell}}&=\frac{1}{n}\sum_{j=1}^n\lay{\delta}{\ell}{_j}
\end{align*}
where we impose the notation of
$$\lay{a}{0}:=x.$$
So we need only give a full characterization of $\lay{\delta}{\ell}.$. To this end, we define recursively starting at layer-$L$ by
\begin{align*}
	\lay{\delta}{L}{^T}&:=d(\L_y)_{\lay{a}{L}}\cdot d\lay{g}{L}_{\lay{z}{L}},\\
	\lay{\delta}{L-1}{^T}&:=\lay{\delta}{L}{^T}\cdot\lay{W}{L}\cdot d\lay{g}{L-1}_{\lay{z}{L-1}},\\
	&\vdots\\
	\lay{\delta}{\ell}{^T}&:=\lay{\delta}{\ell+1}{^T}\lay{W}{\ell+1}d\lay{g}{\ell}_{\lay{z}{\ell}},\\
	&\vdots\\
	\lay{\delta}{1}{^T}&:=\lay{\delta}{2}{^T}\lay{W}{2}d\lay{g}{1}_{\lay{z}{1}},
\end{align*}
as desired.



\subsection{Vectorization in Python}
We implement a neural network with an arbitrary number of layers and nodes, with the $\relu$ function as the activator on all hidden nodes and the sigmoid function on the output layer for binary classification with the log-loss function.

\lstinputlisting[lastline=306]{src/py/deepNeuralNetwork.py}








\subsection{Better Backpropagation}

We consider a neural network of the form
{\tiny
\begin{equation*}
	\begin{tikzcd}
		{}
		&\R^{\lay{n}{1}\times\lay{n}{0}}
		\arrow[d]
		&\R^{\lay{n}{1}}
		\arrow[d]
		&{}
		&\R^{\lay{n}{2}\times\lay{n}{1}}
		\arrow[d]
		&\R^{\lay{n}{2}}
		\arrow[d]
		&{}
		&\R^{\lay{n}{2}}
		\arrow[d]
		&{}
		\\
		\R^{\lay{n}{0}}
		\arrow[r]
		&\boxed{\lay{\phi}{1}}
		\arrow[r]
		&\boxed{\lay{\psi}{1}}
		\arrow[r]
		&\boxed{\lay{G}{1}}
		\arrow[r]
		&\boxed{\lay{\phi}{2}}
		\arrow[r]
		&\boxed{\lay{\psi}{2}}
		\arrow[r]
		&\boxed{\lay{G}{2}}
		\arrow[r]
		&\boxed{\L}
		\arrow[r]
		&\R
	\end{tikzcd}
\end{equation*}
}

where we have the functions:
\begin{enumerate}
	\item $$\lay{G}{\ell}:\R^{\lay{n}{\ell}}\to\R^{\lay{n}{\ell}}$$
is the broadcasting of the activation unit $\lay{g}{\ell}:\R\to\R$.
	\item $$\lay{\phi}{\ell}:\R^{\lay{n}{\ell}\times\lay{n}{\ell-1}}\times\R^{\lay{n}{\ell-1}}\to\R^{\lay{n}{\ell}}$$
	is given by
	$$\lay{\phi}{\ell}(W,x)=Wx.$$
	\item $$\lay{\psi}{\ell}:\R^{\lay{n}{\ell}}\times\R^{\lay{n}{\ell}}\to\R^{\lay{n}{\ell}}$$
	is given by
	$$\lay{\psi}{\ell}(b,x)=x+b.$$
	\item $$\L:\R^{\lay{n}{2}}\times\R^{\lay{n}{2}}\to\R$$
	is the given loss-function.
\end{enumerate}


We now consider back-propagating through the neural network via ``reverse exterior differentiation''.  We represent our various reverse derivatives via the following diagram:

{\tiny
\begin{equation*}
	\begin{tikzcd}
		{}
		&\R^{\lay{n}{1}\times\lay{n}{0}}
		&\R^{\lay{n}{1}}
		&{}
		&\R^{\lay{n}{2}\times\lay{n}{1}}
		&\R^{\lay{n}{2}}
		&{}
		&\R^{\lay{n}{2}}
		\arrow[d]
		&{}
		\\
		\R^{\lay{n}{0}}
		&\boxed{\lay{\phi}{1}}
		\arrow[l, swap, "r"]
		\arrow[u, "r_1"]
		&\boxed{\lay{\psi}{1}}
		\arrow[l, swap, "r"]
		\arrow[u, "\cl{r}_1"]
		&\boxed{\lay{G}{1}}
		\arrow[l, swap, "r"]
		&\boxed{\lay{\phi}{2}}
		\arrow[l, swap, "r"]
		\arrow[u, "r_2"]
		&\boxed{\lay{\psi}{2}}
		\arrow[l, swap, "r"]
		\arrow[u, "\cl{r}_2"]
		&\boxed{\lay{G}{2}}
		\arrow[l, swap, "r"]
		&\boxed{\L}
		\arrow[l, swap, "r"]
		&\R
		\arrow[l]
	\end{tikzcd}
\end{equation*}
}

First, we need to consider our individual derivatives:
\begin{enumerate}
	\item Suppose $G:\R^n\to\R^n$ is the broadcasting of $g:\R\to\R$.  Then for $(x,\xi)\in T\R^n$, we have that
	\begin{align*}
		dG_x(\xi)&=G'(x)\odot\xi\\
		&=\diag(G'(x))\cdot\xi
	\end{align*}
	and for any $\zeta\in T_{G(x)}\R^n$, the reverse derivative is given by
	\begin{align*}
		rG_x(\zeta)&=G'(x)\odot\zeta\\
		&=\diag(G'(x))\cdot\zeta.
	\end{align*}
	
	\item Suppose $\phi:\R^{m\times n}\times\R^n\to\R^m$ is given by $$\phi(A,x)=Ax.$$
	Then we have two differentials to consider:
	\begin{enumerate}
		\item For any $(A,x)\in\R^{m\times n}\times\R^n$ and any $\xi\in T_x\R^n$, we have that
		\begin{align*}
			d\phi_{(A,x)}(\xi)&=A\xi\\
			&=L_A(\xi);
		\end{align*}
		and for any $\zeta\in T_{\phi(A,x)}\R^m$, we have the reverse derivative
		\begin{align*}
			r\phi_{(A,x)}(\zeta)&=A^T\zeta\\
			&=L_{A^T}(\zeta);
		\end{align*}
		where $L_A(B)=AB$, i.e., left-multiplication by $A$.
		
		\item For any $(A,x)\in\R^{m\times n}\times\R^n$ and any $Z\in T_A\R^{m\times n}$ we have that
		\begin{align*}
			d_1\phi_{(A,x)}(Z)&=Zx\\
			&=R_x(Z);
		\end{align*}
		and for any $\zeta\in T_{\phi(A,x)}\R^m$, we have the reverse derivative
		\begin{align*}
		r_1\phi_{(A,x)}(\zeta)&=\zeta x^T\\
		&=R_{x^T}(\zeta);	
		\end{align*}
		where $R_A(B)=BA$, i.e, right-multiplication by $A$.

	\end{enumerate}
	
	\item Suppose $\psi:\R^n\times\R^n\to\R^n$ is given by
	$$\psi(b,x)=x+b.$$
	Then we again have two (identical) differentials to consider:
	\begin{enumerate}
		\item For any $(x,b)\in\R^n\times\R^n$ and any $\xi\in T_x\R^n$, we have that
		$$d\psi_{(b,x)}(\xi)=\xi;$$
		and for any $\zeta\in T_{\psi(b,x)}\R^n$, we have the reverse derivative
		$$r\psi_{(b,x)}(\zeta)=\zeta.$$
		
		\item For any $(x,b)\in\R^n\times\R^n$ and any $\eta\in T_b\R^n$, we have that
		$$d_1\psi_{(b,x)}(\eta)=\eta;$$
		and for any $\zeta\in T_{(\psi(b,x)}\R^n$, we have the reverse derivative
		$$\cl{r}_1\psi_{(b,x)}(\zeta)=\zeta.$$
	\end{enumerate}
\end{enumerate}


\begin{prop}
	Suppose we have the compositional diagram
	\begin{equation*}
		\begin{tikzcd}
			\R^n
			\arrow[r, "f"]
			&\R^m
			\arrow[r, "g"]
			&\R^k
			\arrow[r, "h"]
			&\R^l
		\end{tikzcd}
	\end{equation*}
	and we let $F=h\circ g\circ f:\R^n\to\R^l.$
	Then for any $x\in\R^n$ and any $\zeta\in T_{F(x)}\R^l$, the reverse derivative satisfies
	$$rF_x(\zeta)=rf_x\circ rg_{f(x)}\circ rh_{g(f(x)}(\zeta).$$
\end{prop}

\begin{proof}
For any $\xi\in T_x\R^n$ and any $\zeta\in T_{F(x)}\R^l$, we have by definition
\begin{align*}
	\ip{rF_x(\zeta),\xi}_{\R^n}&=\ip{\zeta,dF_x(\xi)}_{\R^l}\\
	&=\ip{\zeta,dh_{g(f(x))}\circ dg_{f(x)}\circ df_x(\xi)}_{\R^l}\\
	&=\ip{rh_{g(f(x))}(\zeta),dg_{f(x)}\circ df_x(\xi)}_{\R^k}\\
	&=\ip{rg_{f(x)}\circ rh_{g(f(x))}(\zeta),df_x(\xi)}_{\R^m}\\
	&=\ip{rf_x\circ rg_{f(x)}\circ rh_{g(f(x))}(\zeta),\xi}_{\R^n}
\end{align*}	
as desired.
\end{proof}

\begin{lem}
	Suppose $f:\R^{n\times m}\to\R^k$, and for $P\in\R^{n\times m}$, let $R=rf_P$.  Then $R\in\R^k{_n}{^m}$ is rank $(1,2)$-tensor written in coordinates as
	$$R=R_i{^\mu}{_\nu}\frac{\partial}{\partial X^\mu_\nu}\otimes dx^i,$$
	and the components is given by
	$$R_i{^\mu}{_\nu}=\frac{\partial f^i}{\partial X^\nu_\mu}$$
\end{lem}

\begin{proof}
	Considering the basis vectors $\frac{\partial}{\partial X^\nu_\mu}\in T_P\R^{n\times m}$ and $\frac{\partial}{\partial x^i}\in T_{f(P)}\R^k$ we have that
	\begin{align*}
		R_i{^\mu}{_\nu}&=\ip{R\left(\frac{\partial}{\partial x^i}\right), \frac{\partial}{\partial X^\nu_\mu}}_F\\
		&=\ip{\frac{\partial}{\partial x^i},df_P\left(\frac{\partial}{\partial X^\nu_\mu}\right)}_{\R^k}\\
		&=\ip{\frac{\partial}{\partial x^i},\frac{\partial f^\alpha}{\partial X^\nu_\mu}\frac{\partial}{\partial x^\alpha}}_{\R^k}\\
		&=\delta_{i\alpha}\frac{\partial f^\alpha}{\partial X^\nu_\mu},
	\end{align*}
	as desired.
\end{proof}



Returning to our neural network, for each point $(x_j,y_j)$ in our training set, we first let
$$F_j:=\L\circ\lay{G}{2}\circ\lay{\psi}{2}\circ\lay{\phi}{2}\circ\lay{G}{1}\circ\lay{\psi}{1}\circ\lay{\phi}{1},$$
and we have our cost function
$$\J:=\frac{1}{N}\sum_{j=1}^NF_j.$$

We use the following notation for our inputs and outputs of our respective functions:
\begin{itemize}
	\item $$\lay{\phi}{\ell}:(\lay{W}{\ell},\lay{a}{\ell-1}{_j})\mapsto \lay{u}{\ell}{_j},$$
	\item $$\lay{\psi}{\ell}:(\lay{b}{\ell},\lay{u}{\ell}{_j})\mapsto\lay{v}{\ell}{_j},$$
	\item $$\lay{G}{\ell}:\lay{v}{\ell}{_j}\mapsto\lay{a}{\ell}{_j}.$$
\end{itemize}

Let $p=(\lay{W}{1},\lay{b}{1},\lay{W}{2},\lay{b}{2})$ is a point in our parameter space.  Suppose we wish to apply gradient descent with learning rate $\alpha\in T_{\J(p)}\R$, we would define our parameter updates via
\begin{align*}
	\lay{W}{1}&:=\lay{W}{1}-r_1\J_p(\alpha)\\
	\lay{b}{1}&:=\lay{b}{1}-\cl{r}_1\J_p(\alpha)\\
	\lay{W}{2}&:=\lay{W}{2}-r_2\J_p(\alpha)\\
	\lay{b}{2}&:=\lay{b}{2}-\cl{r}_2\J_p(\alpha).
\end{align*}
Moreover, by linearity (and independence of our training data), we see that
$$r\J_p=\frac{1}{N}\sum_{j=1}^Nr(F_j)_p,$$
so we need only calculate the various reverse derivatives of $F_j$.

To this end, we suppress the index $j$ when we're working with the compositional function $F$.  We calculate the reverse derivatives in the order traversed in our back-propagating path along the network.

\begin{enumerate}
	\item $\cl{r}_2\J_p$:
	\begin{align*}
		\cl{r}_2F_p&=\cl{r}_2(\L\circ\lay{G}{2}\circ\lay{\psi}{2})_p\\
		&=\cl{r}_2\lay{\psi}{2}_p\circ r\lay{G}{2}_{\lay{v}{2}}\circ r\L_{\lay{a}{2}}\\
		&=\id\circ r\lay{G}{2}_{\lay{v}{2}}\circ r\L_{\lay{a}{2}}\\
		&=r\lay{G}{2}_{\lay{v}{2}}\circ r\L_{\lay{a}{2}},
	\end{align*}
	and hence
	\begin{align*}
		\cl{r}_2\J_p&=\frac{1}{N}\sum_{j=1}^Nr\lay{G}{2}_{\lay{v}{2}{_j}}\cdot r\L_{\lay{a}{2}{_j}}
	\end{align*}
	
	\item $r_2\J_p$:
	\begin{align*}
		r_2F_p&=r_2(\L\circ\lay{G}{2}\circ\lay{\psi}{2}\circ\lay{\phi}{2})_p\\
		&=r_2\lay{\phi}{2}_p\circ r\lay{\psi}{2}_{\lay{u}{2}}\circ r\lay{G}{2}_{\lay{v}{2}}\circ r\L_{\lay{a}{2}}\\
		&=R_{\lay{a}{1}{^T}}\circ\id\circ r\lay{G}{2}_{\lay{v}{2}}\circ r\L_{\lay{a}{2}}\\
		&=R_{\lay{a}{1}{^T}}\circ r\lay{G}{2}_{\lay{v}{2}}\circ r\L_{\lay{a}{2}},
	\end{align*}
	and hence
	$$r_2\J_p=\frac{1}{N}\sum_{j=1}^NR_{\lay{a}{1}{^T}{_j}}\circ r\lay{G}{2}_{\lay{v}{2}{_j}}\cdot r\L_{\lay{a}{2}{_j}}.$$
	Notice that this is not just a sum after matrix multiplication since we have composition with an operator, namely, $R_{\lay{a}{1}{^T}{_j}}$.  However, since the learning rate $\alpha\in T_{\J(p)}\R\cong\R$, which may pass through the aforementioned linear composition, we conclude that
	\begin{align*}
		r_2\J_p&=\frac{1}{N}\sum_{j=1}^NR_{\lay{a}{1}{^T}{_j}}\circ r\lay{G}{2}_{\lay{v}{2}{_j}}\cdot r\L_{\lay{a}{2}{_j}}\\
		&=\frac{1}{N}\sum_{j=1}^Nr\lay{G}{2}_{\lay{v}{2}{_j}}\cdot r\L_{\lay{a}{2}{_j}}\lay{a}{1}{^T}{_j}.
	\end{align*}
	
	\item $\cl{r}_1\J_p$:
	\begin{align*}
		\cl{r}_1F_p&=\cl{r}_1(\L\circ\lay{G}{2}\circ\lay{\psi}{2}\circ\lay{\phi}{2}\circ\lay{G}{1}\circ\lay{\psi}{1})_p\\
		&=\cl{r}_1\lay{\psi}{1}_p\circ r\lay{G}{1}_{\lay{v}{1}}\circ r\lay{\phi}{2}_{\lay{a}{1}}\circ r\lay{\psi}{2}_{\lay{u}{2}}\circ r\lay{G}{2}_{\lay{v}{2}}\circ r\L_{\lay{a}{2}}\\
		&=\id\circ r\lay{G}{1}_{\lay{v}{1}}\circ L_{\lay{W}{2}{^T}}\circ\id\circ r\lay{G}{2}_{\lay{v}{2}}\circ r\L_{\lay{a}{2}}\\
		&=r\lay{G}{1}_{\lay{v}{1}}\circ L_{\lay{W}{2}{^T}}\circ r\lay{G}{2}_{\lay{v}{2}}\circ r\L_{\lay{a}{2}},
	\end{align*}
	and hence
	\begin{align*}
		\cl{r}_1\J_p&=\frac{1}{N}\sum_{j=1}^Nr\lay{G}{1}_{\lay{v}{1}{_j}}\cdot\lay{W}{2}{^T}\cdot r\lay{G}{2}_{\lay{v}{2}{_j}}\cdot r\L_{\lay{a}{2}{_j}}.
	\end{align*}
	
	\item $r_1\J_p$:
	\begin{align*}
		r_1F_p&=r_1(\L\circ\lay{G}{2}\circ\lay{\psi}{2}\circ\lay{\phi}{2}\circ\lay{G}{1}\circ\lay{\psi}{1}\circ\lay{\phi}{1})_p\\
		&=r_1\lay{\phi}{1}_p\circ r\lay{\psi}{1}_{\lay{u}{1}}\circ r\lay{G}{1}_{\lay{v}{1}}\circ r\lay{\phi}{2}_{\lay{a}{1}}\circ r\lay{\psi}{2}_{\lay{u}{2}}\circ r\lay{G}{2}_{\lay{v}{2}}\circ r\L_{\lay{a}{2}}\\
		&=R_{x^T}\circ\id\circ r\lay{G}{1}_{\lay{v}{1}}\circ L_{\lay{W}{2}{^T}}\circ\id\circ r\lay{G}{2}_{\lay{v}{2}}\circ r\L_{\lay{a}{2}}\\
		&=R_{x^T}\circ r\lay{G}{1}_{\lay{v}{1}}\circ L_{\lay{W}{2}{^T}}\circ r\lay{G}{2}_{\lay{v}{2}}\circ r\L_{\lay{a}{2}},
	\end{align*}
	and hence
	\begin{align*}
		r_1\J_p&=\frac{1}{N}\sum_{j=1}^NR_{x^T}\circ r\lay{G}{1}_{\lay{v}{1}}\cdot\lay{W}{2}{^T}\cdot r\lay{G}{2}_{\lay{v}{2}}\cdot r\L_{\lay{a}{2}}\\
		&=\frac{1}{N}\sum_{j=1}^Nr\lay{G}{1}_{\lay{v}{1}}\cdot\lay{W}{2}{^T}\cdot r\lay{G}{2}_{\lay{v}{2}}\cdot r\L_{\lay{a}{2}}\cdot x^T
	\end{align*}
\end{enumerate}


